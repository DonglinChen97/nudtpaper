\chapter{研究基础与相关技术}

近年来,深度学习一直是计算机领域乃至多学科交叉应用领域的研究热点,不仅在计算机视觉方向有着广泛的应用,在光学、材料和航空航天等领域也出现了相关研究。尤其对于大数据应用场景,基于深度学习的方法已经成为重要的数据挖掘手段。
对于气动优化乃至整个CFD领域而言,深度学习方法在此领域获得发展并取得成功只是时间问题。
为了便于阐述本课题工作,本章首先介绍了CFD相关的基础理论,主要包括建立在宏观层次的Navier-Stokes(N-S)方程和介观层次的格子Boltzmann方法(Lattice Boltzmann Methods,LBM);其次介绍了本文涉及到的深度学习相关的技术和方法,包括基于深度神经网络(Deep Neural Networks,DNN)和图神经网络(Graph  Neural Networks,GNN)的网络架构;最后介绍了本文使用的实验算例。

\section{计算流体动力学基础}
根据对气体在不同尺度上动力学规律的描述,计算流体力学方法可分为三类:宏观、介观和微观。

宏观尺度上,假设流体连续地充满整个空间,流体被假设为连续介质。满足质量守恒、动量守恒以及能量守恒;在数学上,流体可由欧拉方程组、N-S方程组进行描述;在数值计算上,通过各种离散方法将欧拉方程组或N-S方程组离散成各种代数方程。
介观尺度(分子自由程尺度)上,流体被离散为一系列流体粒子;在数学上,流体由统计力学方程描述;在数值计算上,构造符合一定物理规律的演化机制,通过演化得到与物理规律相符的数值结果。
微观尺度上,不再假设流体介质为连续的,通过对每一个分子的运动进行模拟计算,
然后在采用不同的方式进行统计平均,以获得流体的宏观运动规律。因为要对每一个分子的运动进行模拟计算,微观层面的方法往往需要消耗大量的计算资源。
根据与本文研究内容的相关性,本节重点介绍LBM方法和雷诺平均N-S方程。



\subsection{格子Boltzmann方法的基本原理}

\subsubsection{BGK模型}
Boltzmann方程的基本思想是
在任何一个宏观系统中,每一个分子的微观运动都遵循力学规律,因此只要算出大量分子的个别运动就可以确定系统的宏观参数;
求出每一分子处于某种状态下的概率,通过统计的方法得出系统的宏观参数。 
基于以上思想,物理学家 Boltzmann提出方程的三大假设:
\begin{itemize}
	\item[(1)] 分子相互碰撞只考虑二体碰撞,认为3个分子碰撞的概率很小;
	\item[(2)] 各个分子的速度分布是不依赖于另外的分子而独立存在;
	\item[(3)] 外力不影响局部碰撞的动力学行为。 	
\end{itemize}

定义速度分布函数$f$是空间位置矢量$\vec{r}$,分子速度矢量$\vec{\xi}$以及时间$t$的函数。根据$f$的定义有:
\begin{equation}n=\int f(\vec{r}, \vec{\xi}, t) d \xi\end{equation}
\noindent n即为t时刻,$\vec{r}$处单位体积内的分子数。
根据假设,速度分布函数$f$可由两项引起改变,第一项是分子的运动,第二项是分子的碰撞,
先考虑没有碰撞的情况,$m \vec{a}$为作用在每个分子上的外力,m是分子质量,
任意分子,如果在时间间隔$d t$内无碰撞,则分子位置由$\vec{r}$变为$\vec{r}+d \vec{r}$,
速度变为$\vec{\xi}+\vec{a} d t$,则原t时刻,在$d \vec{r} d \vec{\xi}$中的气体分子全部转移到$\vec{r}+d \vec{r}$,$\vec{\xi}+\vec{a} d t$的$d \vec{r} d \vec{\xi}$中,即有
\begin{equation}f(\vec{r}+d \vec{r}, \vec{\xi}+\vec{a} d t, t+d t) d \vec{r} d \vec{\xi}=f(\vec{r}, \vec{\xi}, t) d \vec{r} d \vec{\xi}\end{equation}

\noindent 在$(\vec{r}, \vec{\xi}, t)$处进行Taylor展开,化简有分子的运动对速度分布函数f的影响:

\begin{equation}
\label{运动}
\left(\frac{\partial f}{\partial t}\right)_{\text {运动 }}=-\vec{\xi} \cdot \frac{\partial f}{\partial \vec{r}}-\vec{a} \cdot \frac{\partial f}{\partial \vec{\xi}}\end{equation}

考虑分子间的碰撞,应用刚体碰撞模型,根据碰撞前后动量守恒和能量守恒可得碰撞对速度分布函数的影响:

\begin{equation}
\label{碰撞}
\left(\frac{\partial f_{1}}{\partial t}\right)_{\text {碰撞 }}=\iint\left(f_{1}^{\prime} f_{2}^{\prime}-f_{1} f_{2}\right) d_{D}^{2}|\vec{g}| \cos \theta d \Omega d \vec{\xi}_{2}\end{equation}

\noindent  其中$\vec{g} =  f_{1} - f_{2}$, $f_{1}$,$f_{2}$为碰撞前分子速度,$f_{1}^{\prime}$,$f_{2}^{\prime}$是碰撞后分子速度;$d_{D}$是分子直径,
$d \Omega$表示球面微元在第一个分子的固定角。

综合公式\ref{运动}和\ref{碰撞}有:
\begin{equation}\left(\frac{\partial f}{\partial t}\right)=\left(\frac{\partial f}{\partial t}\right)_{\text {运动 }}+\left(\frac{\partial f}{\partial t}\right)_{\text {碰撞 }}\end{equation}

\noindent 化简即有:
\begin{equation}
\label{Boltzmann方程}
\left(\frac{\partial f}{\partial t}\right)+\vec{\xi} \cdot \frac{\partial f}{\partial \vec{r}}+\vec{a} \cdot \frac{\partial f}{\partial \vec{\xi}}=\iint\left(f_{1}^{\prime} f_{2}^{\prime}-f_{1} f_{2}\right) d_{D}^{2}|\vec{g}| \cos \theta d \Omega d \vec{\xi}_{1}\end{equation}

由于碰撞项计算涉及复杂的非线性积分,所以Boltzmann方程难以求解。因此,Bhatnagar,Gross和Krook提出使用一个简单的算子$\Omega_{f}$替代方程\ref{Boltzmann方程}右边碰撞项,称为BGK近似模型。
最简单的算子可以认为碰撞使f趋于平衡分布$f^{e q}$。设改变率与$\left(f^{e q}-f\right)$成正比,系数为$\nu$,为碰撞频率即弛豫时间的导数${1}/{\tau_{0}}$,则Boltzmann方程简化为:

\begin{equation}\left(\frac{\partial f}{\partial t}\right)+\vec{\xi} \cdot \frac{\partial f}{\partial \vec{r}}+\vec{a} \cdot \frac{\partial f}{\partial \vec{\xi}}=v\left[f^{e q}(\vec{r}, \vec{\xi}, t)-f(\vec{r}, \vec{\xi}, t)\right]\end{equation}


\subsubsection{LBGK模型}
格子Boltzmann方程是BGK方程的进一步离散形式,这一离散形式包括了速度离散、时间离散、空间离散。
对于时间离散,由于微观粒子时刻在做无规则的热运动,因此微观粒子的速度是连续的,其速度方向和大小有无穷个,但粒子的运动并不会显著影响流体的宏观运动。
因此可以将分子速度简化为有限维速度空间,$\left\{\overrightarrow{e_{0}}, \vec{e}_{1}, \ldots, \overrightarrow{e_{N}}\right\}$,N表示速度种类数。
连续的速度分布函数f也被相应离散为$\left\{f_{0}, f_{1}, \ldots, f_{N}\right\}$,
其中$f_{\alpha}=f_{\alpha}\left(\vec{r}, \overrightarrow{e_{\alpha}}, t\right)$,
$\alpha=0,1, \ldots, N$。于是可得离散的Boltzmann方程:

\begin{equation}
\label{速度离散}
\frac{\partial f_{\alpha}}{\partial t}+\overrightarrow{e_{\alpha}} \cdot \nabla f_{\alpha}=-\frac{1}{\tau_{0}}\left(f_{\alpha}-f_{\alpha}^{e q}\right)+F_{\alpha}\end{equation}

\noindent 其中$f_{\alpha}^{e q}$分子局部平衡分布;$F_{\alpha}$为离散速度空间的外力项。
在速度离散的基础上,进一步进行时间离散和空间离散,公式\ref{速度离散}积分,采用矩形法对公式右边项进行逼近有:

\begin{equation}f_{\alpha}\left(\vec{r}+\overrightarrow{e_{\alpha}} \delta_{t}, t+\delta_{t}\right)-f_{\alpha}(\vec{r}, t)=-\frac{1}{\tau}\left[f_{\alpha}(\vec{r}, t)-f_{\alpha}^{e q}(\vec{r}, t)\right]+\delta_{t} F_{\alpha}(\vec{r}, t)\end{equation}


\subsection{雷诺平均N-S(RANS)方程}
流体流动一般由以下三个基本定律来控制:
(1)质量守恒定律;(2)牛顿第二定律;(3)能量守恒定律。基于这三个基本的物理学定理构建的流动模型,将导出一组控制方程。

%,即连续性方程、动量方程和能量方程



正文内容

正文内容



\section{深度学习相关模型介绍}
深度学习是机器学习的重要分支,自2006年提出以来,深度学习理论和技术以及获得了长足的发展,常见的深度学习模型有深度信念网络\cite{深度信念网络},自动编码器\cite{Bengio2013Representation},卷积神经网络\cite{Lecun1998Gradient},递归神经网络\cite{Williams2014A},生成对抗网络\cite{GAN}等。
此外,深度学习在与强化学习、图网络结合方面也非常成功,比较前沿的研究领域有深度强化学习\cite{Deepreinforcementlearning}和图神经网络\cite{2016Semi}等。
深度学习的快速发展为解决气动优化提供了新的思路和方法,
利用深度学习方法提升气动优化效率的核心思想是基于神经网络构建从输入到输出的映射函数,
从而代替或者加速CFD求解器的迭代计算过程,
不同于常规的图像分类任务,深度神经网络需要在大量数据中学习到从给定输入到对应输出的表示方法。

针对流场数据中的欧式空间数据,经过类比分析,我们发现图像处理中的image-to-image\cite{DBLP:conf/miccai/RonnebergerFB15,DBLP:conf/cvpr/LongSD15,isola2017image,CycleGAN2017,DBLP:conf/cvpr/AmodioK19}的转换方法尤其适用于气动优化场景。
对于非欧式空间数据,本文引入了基于图神经网络的架构进行模型训练。
如何将流场数据处理成为深度神经网络可接受的形式将在第三章详细阐述,本节重点介绍三类用于图像回归任务的深度神经网络和图神经网络。
关于深度学习的其他基础理论知识,包括网络基础结构单元,损失函数,优化算法等,可参见文献\cite{dnnsurvey}。

\subsection{卷积自编码网络}
卷积自编码网络(Convolutional Autoencoders)是自编码网络\cite{Bengio2013Representation}的变体。
自编码网络及其变体都有类似的网络结构:编码器和解码器。
传统的自编码器是一种无监督学习算法,数据没有标签,
输入数据$X$经过编码器处理得到输入数据的特征表示z,编码结果经过解码器得到输出$X^{\prime}$,具体过程可以表示为:
\begin{equation}
z=g(X ; \phi) 
\vspace{-0.2cm}
\end{equation}
\begin{equation}
X^{\prime}=f(z ; \theta)
\end{equation}
其中$g(\bullet ; \phi)$和$f(\bullet ; \theta)$分别表示编码器和解码器,$\phi$和$\theta$是相应的参数。

损失函数一般可以定义为输入$X$和输出$X^{\prime}$的最小均方误差(Mean Squared Error,MSE):
\begin{equation}
Loss_{MSE} = \min _{\phi, \theta}\left\|X-X^{\prime}\right\|_{2}^{2}
\end{equation}

一般而言,z的维度远小于输入$X$的维度,网络通过这样编码和解码的方式,实现对输入数据的降维且尽量不损失数据信息。

但是传统的自编码器在处理图片格式数据时,由于采用了全连接操作,忽略了图像中的空间关系,数据切片和数据堆叠会导致信息大量丢失。
为了克服这一缺点,卷积自编码网络采用卷积层来构造自编码器。
图\ref{fig:CAE}是卷积自编码网络的示意图,深色部分代表编码器,通常由卷积层和池化层构成,卷积层负责信息提取,池化层负责空域下采样;
浅色部分代表解码器,由卷积层和上采样层构成,有时也利用逆卷积操作替代卷积和上采样操作进行原始信息的复原。

\begin{figure}[htp]
	\centering
	%\includegraphics[width=0.42\textwidth]{data/MLP.pdf}
	\includegraphics[width=0.92\textwidth]{figures/CAE.pdf}
	\caption{卷积自编码网络示意图}
	\label{fig:CAE}
\end{figure}

逆卷积操作原理如图\ref{fig:conv_dconv}所示,逆卷积操作可以看出是常规卷积操作的逆过程,不同点在于,为了还原原始输入的尺寸,通常需要进行填充(padding)操作(如图\ref{fig:dconv}中的空白区域)。


\begin{figure}[htb]
	\centering
	\subfloat[卷积操作]{\label{fig:conv}\includegraphics[width=0.42\textwidth]{figures/conv.png}} \qquad
	\subfloat[逆卷积操作]{\label{fig:dconv}\includegraphics[width=0.38\textwidth]{figures/dconv.png}} 
	\caption{逆卷积操作原理}
	\label{fig:conv_dconv}
\end{figure}

%\begin{figure}[htp]
%	\centering
%	%\includegraphics[width=0.42\textwidth]{data/MLP.pdf}
%	subfigure[卷积操作]{\label{fig:conv}\includegraphics[width=0.42\textwidth]{figures/conv.png}}
%	subfigure[逆卷积操作]{\label{fig:dconv}\includegraphics[width=0.42\textwidth]{figures/dconv.png}}
%	\caption{逆卷积操作原理}
%	\label{fig:conv_dconv}
%\end{figure}


对于无监督学习,在卷积神经网络的编码器和解码器的衔接处,利用全连接层,也可以提取带图像数据特征表示;在进行有监督学习时,网络更关注解码器的输出$Y^{\prime}$而不是z,损失函数转化为:
\begin{equation}
Loss_{MSE} = \min _{\phi, \theta}\left\|Y-Y^{\prime}\right\|_{2}^{2}
\end{equation}
从而可以利用卷积神经网络进行有监督学习任务,在气动流场模拟领域已有基于卷积自编码网络开展的工作\cite{DBLP:conf/kdd/GuoLI16}。



\subsection{基于深度学习的图像分割模型}
图像分割一直是计算机视觉领域的难题,也是该领域的研究热点。
所谓图像分割是指根据灰度、彩色、空间纹理、几何形状等特征把图像划分成若干个互不相交的区域,使得这些特征在同一区域内表现出一致性或相似性,而在不同区域间表现出明显的不同。
传统的方法有基于阈值的分割方法;基于区域的图像分割方法;基于边缘检测的分割方法;基于小波分析和小波变换的分割方法;基于遗传算法的分割方法等\cite{图像分割方法综述}。

近年来,深度学习方法开始应用到图像分割,通过搭建神经网络,对训练样本进行有监督学习,得到图形分割的模型。根据分割应用任务不同,图像分割分可分为普通分割、语义分割和实例分割。其中:普通分割是指对分属不同区域的像素点进行分类;语义分割是在分类的基础上识别出每一块区域的语义;实例分割在在语义分割的基础上,进一步对每个识别目标进行编号。本节重点介两种经典的语义分割网络:全卷积网络和U-net网络,分别适用于自然图像分割和医疗图像分割。


\subsubsection{全卷积网络}
2015年Long等人提出的全卷积网络(Fully Convolutional Networks,FCN)用于图像语义分割\cite{Long2015Fully}。自从提出后,FCN已经成为语义分割的基本框架,后续算法都借鉴了该框架的思想。

\begin{figure}[htp]
	\centering
	%\includegraphics[width=0.42\textwidth]{data/MLP.pdf}
	\includegraphics[width=0.88\textwidth]{figures/FCN.png}
	\caption{FCN网络结构图}
	\label{fig:FCN}
\end{figure}

如图\ref{fig:FCN}所示,FCN参考了图像分类网络中的VGG16\cite{2014Very}网络架构。图像分类网络只能对整个图片进行分类而不能识别每个像素点的类别。
为了实现逐像素分类的目的,FCN用卷积层替换了VGG网络中的全连接层,最后利用逆卷积的上采样方法将特征图恢复成原始图片大小,从对整张图片的稀疏分类转换成对每个像素点进行密集分类(dense prediction),达到图像分割的目的。

FCN的另一个特点是利用了全局信息和局部信息。经过多次卷积和池化操作以后,得到的图像越来越小,分辨率越来越低,最后产生了高维特征图。如果直接对进行上采样至原始图片大小,会产生模糊的分割结果。为了解决这一问题,FCN使用了如图\ref{fig:FCN_skip}所示的skip layer的方法。

对于FCN-32s,高层得到的粗糙层(conv7)进行32倍上采样操作,再对每个点进行softmax逻辑回归处理,得到每一个像素点的分类。

对于FCN-16s,先将conv7的结果进行2倍上采样,再将上采样结果与浅层的精细层(pool4)进行逐点相加,最后进行16倍的上采样操作得到与原始输入尺寸相同的图像分割结果。

对于FCN-8s,先将conv7的结果进行4倍上采样,将pool4的结果进行2倍上采样,再将上采样结果与更浅层的精细层(pool3)进行逐点相加,最后进行8倍上采样操作和softmax处理。

通过skip layer的方法,融合多层特征图,有效整合了粗粒度的语义信息和细粒度的位置信息,有利于提高分割准确性。
尽管8倍上采样的FCN的分割效果已经有了很大提升,但是结果还是比较模糊,对分割区域边界不敏感;此外,对每个像素点单独进行分类,没有充分考虑图像的空间上的联系。

\begin{figure}[htp]
	\centering
	%\includegraphics[width=0.42\textwidth]{data/MLP.pdf}
	\includegraphics[width=0.88\textwidth]{figures/FCN_skip.png}
	\caption{FCN采用的skip layer方法}
	\label{fig:FCN_skip}
\end{figure}

\subsubsection{U-net网络}
2015年,Ronneberger等人\cite{DBLP:conf/miccai/RonnebergerFB15}提出了U-net网络结构,U-net是基于FCN的一种语义分割网络,适用于做医学图像的分割。U-net修改并扩展了FCN网络结构,使它在使用少量的数据进行训练的情况下获得精确的分割结果。
其主要思想是在下采样网络的后面补充一个对称的上采样网络,多个上采样层增加了网络的参数和表示能力,有利于提升输出结果的分辨率。
对称的网络结构形似英文字母“U”,所以被称为U-net。

U-net网络结构如图\ref{fig:unet}所示:
蓝/白色框表示特征图;蓝色箭头表示3x3卷积层,用于特征提取;灰色箭头表示 skip-connection,用于特征融合;红色箭头表示池化层,用于降低维度;绿色箭头表示上采样 upsample,用于恢复维度;青色箭头表示1x1卷积,用于输出结果。其中灰色箭头中的copy就是atenate操作,将深层的特征图和浅层的特征图在通道方向拼接;crop是为了让两者的长宽一致,保证拼接后的特征图大小一致。


\begin{figure}[htp]
	\centering
	%\includegraphics[width=0.42\textwidth]{data/MLP.pdf}
	\includegraphics[width=0.88\textwidth]{figures/unet.png}
	\caption{U-net网络结构图\cite{DBLP:conf/miccai/RonnebergerFB15}}
	\label{fig:unet}
\end{figure}

U-net主体结构包括一个捕获上下文信息的收缩路径和一个允许精确定位的对称拓展路径:
收缩部分和扩展部分都有4个采样层。这种架构延续了Encoder-Decoder的思想。
Encoder由卷积操作和下采样操作组成,文中所用的卷积结构统一为3x3的卷积核,填充为 0,步长为1;Decoder部分采用的上采样的方式与FCN中的反卷积不同,为双线性插值。
此外,为了更好融合位置信息和语义信息,U-net拓展了FCN中skip layer的思想,在“U”形网络的对称部分添加了skip connection。与FCN的加操作不同,U-net使用了叠操作(concatenation)增加了特征的厚度,保留了更多浅层的位置信息,这使得上采样层可以在浅层特征与深层特征在训练时进行自适应选择,这对语义分割任务来说更有优势。

由于在医疗图像分割上取得巨大成功,许多研究者针对不同的图像分割任务对U-net进行了改进,产生了许多U-net变体包括V-Net\cite{2016V}、UNet++\cite{unet++}、U-NetPlus\cite{unetplus}和3D U2-Net\cite{20193D}等,提升了模型的推理速度和精度。


\subsection{Pix2Pix网络模型}
对于气动流场预测,核心问题是实现从输入到输出的映射。
除了前文提到的自编码网络和图像分割网络,基于生成对抗网络(Generative Adversarial Network, GAN)的图像风格迁移模型也在解决像素到像素的映射问题上展示出的强大潜力。
GAN在图像生成、风格迁移、超分辨重建等领域已经得到了广泛的应用,
本节我们重点介绍Pix2Pix网络模型\cite{isola2017image}。

Pix2Pix网络基于条件生成对抗网络(Conditional Generative Adversarial Network, cGAN)\cite{cGAN},通过添加条件约束信息来指导完成图像转换任务,比如从标签图合成相片,从线稿图重构对象,给图片上色等。
和传统的GAN网络类似,在Pix2Pix模型训练过程中,迭代地训练生成器和判别器,生成器尽可能生成接近真实的样本,企图“欺骗”判别器;判别器尽可能识别出真实的样本和生成的样本,获得更高的得分。这样的对抗训练过程类似博弈游戏,随着训练的进行,生成器和判别器的能力不断提升直到达到令人满意的效果。
由于添加了约束条件,Pix2Pix网络的输入不再是普通GAN网络中的随机变量,其网络结构示意图
如图\ref{fig:cgan}所示:

\begin{figure}[htp]
	\centering
	%\includegraphics[width=0.42\textwidth]{data/MLP.pdf}
	\includegraphics[width=0.88\textwidth]{figures/cGAN.png}
	\caption{Pix2Pix网络结构示意图}
	\label{fig:cgan}
\end{figure}

其中x是条件输入,y是对应的标签,每个输入x唯一对应一个标签y。
在训练生成器时,x进过生成器$G$得到生成图像$G(x)$;
在训练判别器时,输入x和对应生成图像$G(x)$或标签y通过通道维度的叠操作进行拼接,一同送入判别器,判别器会输出概率值,表示是否为一对真实样本。



Pix2Pix网络利用类似U-net的结构代替自编码网络作为生成器,
在输入和输出之间存在很多可以共享的低级信息,采用skip-connection的结构有利于传递这些底层信息和重构图像。
对于判别器网络,Pix2Pix提出了PatchGAN架构。
传统GAN判别器通常对生成样本整体进行判断,对于图片而言,直接输出整张图片是真实样本的概率。而图像转换任务中关注像素到像素的转换效果,所以在这里提出了分块判断的算法,在图像的每个块上去判断是否为真,输出平均预测结果。


条件GAN采用的损失函数通常为:
\begin{equation}
\begin{aligned}
\mathcal{L}_{c G A N}(G, D)= &\mathbb{E}_{x,y}[\log D(x,y)]+\\
&\mathbb{E}_{x,z }[\log (1-D(x, G(x,z))]
\end{aligned}
\end{equation}

\begin{equation}
\begin{aligned}
\mathcal{L}_{c G A N}(G)= \mathbb{E}_{x,z}[\log (1-D(x, G(x, z))]
\end{aligned}
\end{equation}

其中z为输入随机变量,在Pix2Pix网络输入仅为x。
此外,为了保证像素级低频信息的预测精度,Pix2Pix网络的损失函数还引入了$L_1$损失项:

\begin{equation}
\begin{aligned}
\mathcal{L}_{L 1}(G)=\mathbb{E}_{x_1, x_2, y}\left[\|y-G(x_1, x_2)\|_{1}\right]
\end{aligned}
\end{equation}

考虑到训练的最终目标是获得一个性能良好的生成器,同时生成器和判别器进行着对抗的训练,所以训练的最终目标是:
\begin{equation}
\begin{aligned}
G^{*}=\arg \min _{G} \max _{D} \mathcal{L}_{c G A N}(G, D)+\lambda \mathcal{L}_{L 1}(G)
\end{aligned}
\end{equation}
\noindent 其中$\lambda$是$L_1$损失函数项的权重系数。


\subsection{图神经网络}



\section{实验算例介绍}
正文内容

正文内容

正文内容

正文内容

正文内容

正文内容

\subsection{2D不可压层流固体外部流场}
正文内容

\subsection{2D不可压有粘翼型外部流场}
正文内容

正文内容




\section{本章小结}
