
%%% Local Variables:
%%% mode: latex
%%% TeX-master: "../main"
%%% End:

\begin{ack}
行文至此,硕士研究生学习生活已经接近尾声。在毕设论文完成之际,谨向这两年半时间里在生活学习等各方面给予我支持与帮助的人表示最衷心的感谢!

首先,感谢两年半前的自己,是你之前的努力换来了打开国防科大研究生生活大门的钥匙,让我有幸在一个高规格、高水平和高层次的平台继续深造。

感谢我的硕士指导老师邓小刚老师!邓老师学识渊博,治学严谨。
在毕设选题和研究内容上进行了建设性的指导,在课题研究方向和研究方法提出了宝贵的建议。
邓老师对待学术研究态度严谨、一丝不苟的品格也一直激励着我不断在研究领域进行探索,跟随邓老师进行学术研究的两年半获益匪浅。

由衷感谢课题组的徐传福老师!从硕士入学到即将毕业徐老师一直悉心指导和周密安排,无论生活中还是实验上遇到难处,徐老师都会积极开导,帮助解决,每次与之交流都收获颇丰。
在组会讨论时,徐老师都会认真倾听汇报并给出针对性建议,及时帮助我解决课题研究中遇到的困难。徐老师是我硕士生涯的引路人,一步一步指引我在学术研究上不断取得突破。

感谢方建滨师兄和高翔师兄!两位师兄分别在我硕士前后阶段对我的研究工作进行了细致耐心的指导,传授了许多宝贵的学习经验,让我少走了不少弯路。两位优秀的师兄有着严谨的治学态度和良好的生活学习习惯,一直是我学习看齐的榜样!

感谢课题组陈世钊师兄,帮助解决了生活和学习中很多难题,从钊哥身上学习到很多为人处世的方法!感谢课题组熊敏、程彬、郭宁波、林玉、刘舒、苗秋成、张海红、顾善植等师兄师姐,在学习和生活上提供很多指导!感谢课题组廖海翔、郭睿、史玮三位同级,在日常工作和学习上大家相互配合相互促进!感谢课题组王啸宇、高婉蓉、李文强、朱东等师弟师妹,有了你们实验室的学习生活不再单调枯燥。

感谢203小分队成员郭凌超、黄永钦和李晨,两年半以来我们同甘共苦,从你们身上我学到很多,是你们解决我生活的后顾之忧,为生活增添了很多乐趣。

感谢我的父母,一直无条件支持我、鼓励我、爱我,为我提供了温馨幸福的生活条件,解决了我学习和工作的后顾之忧。

最后感谢我的女朋友张燕平,谢谢你的一直陪伴和支持,牺牲和理解,包容和爱。


\end{ack}
