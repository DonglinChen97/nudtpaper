\chapter{基于GCN网络的流场模拟加速方法}

本章介绍了基于图神经网络的气动流场模拟加速方法,
首先利用CFD网格生成软件在算例几何模型上生成非结构网格,
基于网格单元拓扑连接信息、边界条件和初始条件提取图神经网络训练的输入数据;
利用CFD求解器获取图神经网络训练的真值。
然后基于图卷积神经网络设计了气动流场模拟加速模型,在模型训练完成后,
使用CFD求解器对气动流场模拟加速模型的输出结果进行优化。
最后在测试集上比较了气动流场模拟加速方法与传统CFD数值模拟方法的气动流场模拟效率。

\section{动机分析}

利用深度学习方法对气动流场进行预测的主要思想是将气动流场模拟问题转化为图像回归预测问题,
因此需要将流场数据转化为矩阵的形式。
在处理流场中非结构数据时,则需要通过采样的方式将数据转化为结构规则的笛卡尔网格形式,
这样就会导致预测的结果的不能真实的反映物理量在局部流场的分布,比如在进行传统CFD模拟,
边界层的网格密度远大于远场,因为边界层的速度和压力等物理量变化更加剧烈,
使用均一的粗粒度的网格进行模拟根本不能得到准确的结果甚至导致计算不收敛。
此外,流场预测结果由深度学习模型输出,在复杂流动条件下,无法保证深度学习模型能够对流体运动规律进行准确学习、输出符合流体运动规律的预测结果。

基于以上考虑,本文在利用图神经网络进行流场模拟方面进行了探索,对气动流场模拟的效率和预测结果的有效性进行了权衡,
提出了基于图神经网络的流场模拟加速方法。
与基于传统卷积神经网络的流场预测方法相比,该方法主要有两点不同:
1)流场数据表示方法通用性更强。
无论是结构网格还是非结构网格,利用图能够灵活地对网格单元的拓扑结构进行表示,
边界条件和初始条件可以处理为节点特征向量。
由于图结构利用网格对流场域进行了全尺寸模拟,不需要利用采样方法将流场表示为笛卡尔网格,
所以最大程度保留了原始流场数据中的信息,一些控制流体流动的全局变量也可以在图神经网络训练时用于每一个图节点上。
2)从根本上保证流场模拟结果的有效性。
利用CFD求解器对深度学习模型的预测结果进行优化,将计算收敛的结果作为最终的流场预测结果。
虽然深度学习模型在此方法中只是作为加速模块,部分代替了CFD求解器的工作,
导致气动流场模拟效率低于基于深度学习的流场预测方法,
但由于气动流场模拟结果最终由CFD求解器输出,因此预测结果满足计算收敛条件和流体流动物理规律。

\section{数据表示及训练集生成}


\subsection{网格文件预处理}

\subsection{基于OpenFoam生成训练集}
正文内容
正文内容
正文内容



\section{网络架构与实现}
正文内容

正文内容



正文内容

正文内容

正文内容

正文内容

\section{实验设置和结果分析}

层数、隐藏节点数、其他的超参数

\subsection{参数设置}



\subsection{实验结果与分析}



\section{本章小结}

