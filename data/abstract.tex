\begin{cabstract}

气动优化设计是航空航天飞行器设计等应用的重要组成部分,随着飞行器产业的发展和应对复杂现实环境需求的提升,
气动优化往往涉及许多相互交织影响的因素,容易导致设计优化空间“爆炸”。全阶的计算流体力学(CFD)方法能够针对特定流动状态进行高精度的模拟,但往往耗时较长难以进行全面的设计空间探索;虽然基于传统机器学习方法的优化技术在构建代理模型、降阶模型等方面有广泛的应用,
但该方法多适用于简单气动优化问题,具有一定的局限性。
近年来,深度学习相关研究不断深入并推动诸多应用领域进一步发展,为提升气动优化效率,
构建高效、准确的气动评估方法提供了新的途径。本文主要研究内容如下:

\begin{itemize}
	\item[(1)] 通过对比研究气动流场模拟任务和图像回归预测任务的相似性和不同点,本文提出了一种基于深度卷积网络的气动流场预测方法。首先,针对流场数据的表示问题,提出基于笛卡尔网络的流场数据表示方法,将流场数据转化为深度神经网络可以接受的形式。
	其次,针对边界层等区域流场物理量变化大、包含重要信息多特点,本文提出了嵌入注意力模块的深度卷积网络,能够有效学习该区域流动规律,提升模型预测准确率;基于流体运动遵循的质量守恒定律和动量守恒定律,设计了物理约束的损失函数,有效保证了深度学习预测模型预测结果和CFD求解器的物理一致性。为了全面的评估气动流场预测模型的性能,本文定义了新的性能指标,在测试集上的对比试验结果表明该方法能有效提升气动流场模拟效率,同时将预测误差控制在5\%以内。
	\item[(2)] 针对深度卷积网络难以处理非欧式空间流场数据的问题,本文提出了一种基于图神经网络的气动流场模拟加速方法。
	基于图结构对流场数据进行表示,能够有效减少数据转换过程中的信息丢失,实现流场几何外形、边界条件和流动控制变量的有效融合;
	基于图卷积网络的流模拟加速模块能够提取流场数据特征,为CFD求解器提供接近收敛状态的初始场,减少CFD迭代计算的时间。
	实验结果表明该方法能够在加速气动流场模拟的同时,从根本上保证了模拟结果的有效性。
	
\end{itemize}

本文针对飞行器气动优化设计中的典型问题,基于深度学习技术和方法,提出针对流场数据的表示方法,搭建气动流场预测模型和气动流场模拟加速模型,通过嵌入物理约束和融合传统CFD方法进一步优化模型预测结果,有效提升气动流场模拟效率,为基于深度学习的智能气动流场和气动性能预测技术在复杂飞行器气动优化设计中的应用奠定基础。

\end{cabstract}
\ckeywords{气动优化; 计算流体力学; 深度学习; 图神经网络; 物理一致性}



\begin{eabstract}
National University of Defense Technology is a comprehensive national key university based in Changsha, %
Hunan Province, China. It is under the dual supervision of the Ministry of National Defense %
and the Ministry of Education, designated for Project 211 and Project 985, %
the two national plans for facilitating the development of Chinese higher education. %

NUDT was originally founded in 1953 as the Military Academy of Engineering in Harbin of Heilongjiang Province. %
In 1970 the Academy of Engineering moved southwards to Changsha and was renamed Changsha Institute of Technology.%
 The Institute changed its name to National University of Defense Technology in 1978.

\end{eabstract}
\ekeywords{Aerodynamic Optimization; Computational Fluid Dynamics; Deep Learning; Graph  Neural  Network; Physical Consistency}

