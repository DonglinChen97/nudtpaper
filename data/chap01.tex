\chapter{绪论}
CFD数值模拟在航空航天飞行器气动优化设计等应用中发挥着重要作用,
但是全阶CFD模拟经济、时间成本高昂,限制了设计人员进行全面、快速的设计空间探索和设计迭代。为了以更省时省力、又快又好地获得最优解,机器学习等智能算法和技术在气动优化的寻优算法及构建代理模型、降阶模型等方面有广泛的应用,但是传统机器学习方法适用于处理低维数据,难以推广到复杂的CFD应用场景。随着人工智能领域的快速发展,以深度学习为代表的智能方法和技术在处理高维复杂数据上展现了强大的学习能力,从而为快速、精确气动评估提供了新思路和新方法。



\section{研究背景及意义}
计算流体力学( Computational Fluid Dynamics,简称 CFD )作为了解和探索流体运动的手段,在航空航天,交通运输,石油勘探,天气预测,水利工程和石油化工等工程领域发挥着及其重要的作用。近年来,随着计算机性能的提升,CFD的应用前景进一步扩大,渗透到生活的方方面面。小到吸管设计,大到汽车制造,生活中随处可见CFD的身影。在2020年新冠疫情期间,CFD研究者通过对喷嚏飞沫的流动状态进行研究,得出喷嚏飞沫可以漂浮8米远的结论,为疫情防控做出了重要的贡献\cite{JAMA-喷嚏}。

复杂飞行器气动优化设计是CFD的重要应用领域,以增升减阻为目标的气动优化是飞行器设计的重要组成部分。
如文献\cite{增升减阻ep}指出,对于某民航客机,起飞升阻比每提高1\%,可增加14个乘客;
着陆最大升力系数每增加1\%,则可增加22个乘客。
由此可见,提升气动优化效率,实现快速、精准的气动评估对于推动飞行器设计发展具有重要意义。
尽管传统的全阶CFD数值模拟可针对特定状态获得较高精度的评估结果,但是时间、经济等成本巨大。
为了以更省时省力、又快又好地获得最优解,许多研究者在优化设计时对原始模型进行简化处理,包括采用更为简单的代理模型\cite{代理模型}和降阶模型\cite{降阶模型},代替复杂的CFD评估过程,减少计算开销;引入基于机器学习的智能方法和技术等。因此,在保证模拟精度的基础上,不断地提升气动模拟的效率是数值模拟领域现阶段的研究热点,具有极大的实际应用价值。然而加速和优化气动评估目前面临着以下挑战:

\vspace{-0.2cm}
\begin{itemize}
	\item[(1)] 气动优化往往涉及许多相互交织影响的因素。以常见的超临界机翼优化为例,除需考虑巡航升阻力系数、升阻比、力矩系数等设
	计点性能外,抖振、阻力发散等非设计点动态特性也需考虑;同时,优化还存在一些必要约束,如机翼厚度、油箱容积、前后缘装置等。海量设计参数容易导致设计空间“爆炸”,仅依靠传统的CFD数值模拟极大地限制了对复杂飞行器进行全面的设计空间探索。
	\item[(2)] 机器学习等智能算法和技术,通过部分或者全部代替复杂的CFD评估过程,在原始模型进行简化处理方面有广泛的应用。
	然而,大部分机器学习技术借助于发展成熟机器学习方法和技术,属于浅层学习方法,随着CFD所模拟的问题越加的复杂,预测精度和应用范围相对有限。
	此外,传统机器学习算法复杂度将随着样本数量和模
	型精度的提高呈指数级增长,目前应用范围多限于一些容易获得训练样本的二维简单优化
	问题。

\end{itemize}

自2006年Hinton等提出深度信念网络\cite{深度信念网络}以来,在GPU、TPU等高性能计算机硬件的助力下,以深度学习为代表的智能方法和技术迅速发展,成功应用于图像分类与识别,医疗诊断,视频预测,自然语言处理等诸多领域。
在学术界,
由于采用了复杂和更深层次的模型结构,深度学习模型更善于从数据中提取特征,而不是依靠人工构建特征,极大提升了归纳能力,
能够自主进行特征选择,可以从大量的候选特征中剔除无用特征再进行回归和分类,具备
深层次学习能力,尤其适合于归纳、分析高维、时空相关的流场数据,展现出广阔的应用前景。

\FIXME{增加内容}

一方面,深度学习方法拥有巨大的潜力解决CFD领域所面临的问题;另一方面,
基于深度学习的气动评估及其在气动优化中的应用尚处于起步阶段,已有应用多为二
维简单外形算例,多参考深度学习在其他领域应用较为成熟的方法技术,尤其是深度卷积
神经网络在计算机视觉领域的研究成果,深度学习还没有与空气动力学实现交叉融合。
因此,研究基于深度学习的气动优化技术对于解决气动优化领域所面临的问题和满足设计空间探索的实际需求拥有重大意义。



\subsection{(1.1.1 题目)}
正文内容

正文内容

\begin{figure}[htp]
\centering
\includegraphics{picmain}
\caption{图 1.1 名称}
\end{figure}

\subsubsection{(1.1.1.1 题目)}
正文内容

正文内容

正文内容

\subsubsection{(1.1.1.2 题目)}
正文内容

正文内容

正文内容

\subsection{(1.1.2 题目)}
正文内容

正文内容

\begin{figure}[htp]
\centering
\includegraphics{picmain}
\caption{图 1.2 名称}
\end{figure}

\section{研究现状}
正文内容

正文内容



正文内容

正文内容

正文内容

正文内容

\section{主要工作和创新点}
正文内容

正文内容

正文内容

正文内容

正文内容

正文内容

\section{论文结构}

