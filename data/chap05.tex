\chapter{总结与展望}


\section{本文主要工作总结}

气动优化设计是新型飞行器设计的重要组成部分,飞行器气动优化往往涉及许多相互交织影响的因素,
设计人员在进行设计空间探索时需要对不同工况下飞行器的气动性能进行评估。
虽然全阶的CFD方法可针对特定状态获得较高精度的评估结果,但进行迭代设计时间、经济成本巨大。
本文面向复杂飞行器气动优化设计等应用对快速、精确气动评估的迫切需求,针对汽车和翼型等气动优化设计典型问题,
提出了基于深度学习的气动流场预测方法和模拟加速方法,提高了气动模拟的效率。
本文主要内容和创新性成果如下:

\begin{itemize}
	\item[(1)] 针对传统CFD方法进行流场模拟效率低的问题,
	提出一种基于深度卷积网络的端到端流场预测方法。
	首先针对流场数据表示,研究了基于笛卡尔网络的流场数据表示方法。
	然后,提出了基于U-net网络的气动流场预测模型\textsc{FlowDNN},
	针对流体运动特点,设计新型深度网络架构;
	基于流体流动的物理规律,提出嵌入物理约束的损失函数;
	利用注意力机制提升在边界层等RoI区域的预测准确率;
	引入神经网络剪枝,精简模型结构,进一步提升预测效率。
	最后定义了新的流场预测结果评价指标,与传统CFD求解器和三个基线模型进行了全面的性能比较,
	在测试集上的对比试验结果表明该方法能有效提升气动流场模拟效率。
	
	\item[(2)] 针对深度卷积网络难以处理非欧式空间流场数据的问题,
	本文利用图数据结构对流场数据进行更加通用的表示,提出了一种基于图神经网络的气动流场模拟加速方法。
	基于图结构对流场数据进行表示,能够最大程度保留了原始流场数据中的信息,
	实现流场几何外形、边界条件和流动控制变量的有效融合;
	基于图卷积网络的流模拟加速模块部分代替了CFD求解器的工作,利用CFD求解器对模型预测结果进行优化,保证预测结果的物理一致性。
	实验结果表明该方法能够在加速气动流场模拟的同时保证了模拟结果的有效性。
	
\end{itemize}

本课题针对飞行器翼型气动优化设计等典型应用领域快速精确气动评估需求,建立气动流场的深度学习预测模型,
设计嵌入物理模型提升预测模型的精度和预测结果的可信度,
推动深度学习算法在解决气动流场预测、模拟加速和优化设计等方面的迁移和发展。
在流场数据表示方面,本文基于图数据结构提出了更加通用合理的流场数据表示方法,利用图神经网络提高了流场模拟效率,
丰富了深度学习方法和技术在CFD领域的应用场景;此外,本文融合深度学习方法和CFD方法对气动流场进行模拟,保证了模拟结果的物理一致性,
为基于深度学习技术解决CFD领域实际应用问题奠定了基础。

\section{研究展望}

本课题下一步的研究方向如下:
\begin{itemize}
	\item[(1)] 本文基于图神经网络的气动流场模拟加速模型比较简单,与CFD领域的联系还不够紧密,
	可以基于CFD求解器中网格离散计算方法设计专用于气动流场模拟的图卷积算子,借鉴成熟的图神经网络优化算法构建更加高效的网络模型。
	
	\item[(2)] 现阶段工作还没有充分利用CFD数据提供的先验知识,神经网络输入的信息有限,
	可以利用CFD方法中的粗细网络结合思想,基于粗网络的模拟结果对细网格的流场进行加速模拟。
	
	\item[(3)] 现阶段工作深度学习网络和CFD求解器还没有充分的融合,
	可以设计可导的CFD求解器参与神经网络训练,在保证模拟结果的同时,增加气动流场模拟模型的泛化能力。	

	\item[(4)] 本文的实验算例设置还比较简单,基于图卷积网络的气动流场加速模型只能在固定网格的基础上对不同流动条件下的气动流场进行模拟加速,
	未来可将现阶段研究成果拓展至更加复杂的气动优化场景,同时基于前沿的图神经网络相关方法和技术,探索实现不同网格条件下的流场模拟加速方法。
	
\end{itemize}
	
	